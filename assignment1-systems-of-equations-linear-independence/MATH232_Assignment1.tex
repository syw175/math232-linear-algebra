\documentclass[10pt]{article}
\usepackage[paper=letterpaper,margin=2cm]{geometry}
\usepackage{amsmath}
\usepackage{amssymb}
\usepackage{amsfonts}
\usepackage{newtxtext, newtxmath}
\usepackage{enumitem}
\usepackage{titling}
\usepackage{fancyhdr}
\usepackage[colorlinks=true]{hyperref}

\setlength{\droptitle}{-10em}

\fancyhf{}
\fancyhead[L]{MATH 232}
\fancyhead[R]{Steven Wong, 301337727}
\renewcommand\headrulewidth{0pt}
\pagestyle{fancy}

\begin{document}

\noindent\makebox[\textwidth][c]{\Large\bfseries Assignment 1}
\normalsize
\begin{enumerate}[leftmargin=\labelsep]
    \item[1a.)] I generated 100 random $n\times n$ sized matrices with $8 \leq n \leq 20$ of real numbers; checking whether their RREF's was equal to the $n$ identity matrix. Of 100 test cases, 100/100 evaluated true with an unique solution. 
    
        $$ \text{Exception: } \begin{bmatrix}\label{eq1}
        1 & 2 & 3 & 1\\
        1 & 2 & 3 & 2\\
        1 & 2 & 3 & 2
        \end{bmatrix} 
        \to
        \begin{bmatrix}
        1 & 2 & 3 & 0\\
        0 & 0 & 0 & 1\\
        0 & 0 & 0 & 0
        \end{bmatrix}
        $$
    
    \item[1.b)] Similar methodology from Part(A) with the following test for inconsistency: If the rank of the coefficient matrix is less than the rank of the augmented matrix, the system of equations is inconsistent. Of 100 test cases, 100/100 evaluated true with no solutions. 
    
        $$ \text{Exception: } \begin{bmatrix}\label{eq1}
        1 & 2 & 3 & 1\\
        2 & 4 & 6 & 2\\
        3 & 3 & 4 & 3\\
        10 & 10 & 12 & 4
        \end{bmatrix} 
        \to
        \begin{bmatrix}
        1 & 0 & 0 & 2.5\\
        0 & 1 & 0 & -7.5\\
        0 & 0 & 1 & 4.5\\
        0 & 0 & 0 & 0
        \end{bmatrix}$$

    \item[1.c)] Similar methodology from Part(B) with the following test for many solutions: If the system was consistent, then I verified that $n -$ rank(coefficient) $> 0$ which meant many solutions exist. Of 100 test cases, 100/100 evaluated true with many solutions.

       $$\text{Exception: } \begin{bmatrix}\label{eq1}
        1 & 2 & 3 & 4 & 1\\
        5 & 6 & 7 & 8 & 2\\
        9 & 10 & 11 & 12 & 4
        \end{bmatrix} 
        \to
        \begin{bmatrix}
        1 & 0 & -1 & -2 & 0 \\
        0 & 1 & 2 & 3 & 0 \\
        0 & 0 & 0 & 0 & 1\\
        \end{bmatrix}$$

    \item[2.a)] \textbf{Show $B_1$ is a linearly dependent set and Theorem 1.2.2: } The set of vectors in $B_1$ is linearly dependent if there is a non-trivial solution to $c_1w_1 + c_2w_2 + c_3w_3 + c_4w_4 + c_5w_5 + c_6w_6 = 0$. I expressed these equations in $c_i$s into an augmented matrix form and solved it.
    The resulting RREF indicates that $w_5$ and $w_6$ are linearly dependent as leading 1's do not exist for those columns.
    
        $$B_1 RREF: \begin{bmatrix}
        1 & 0 & 0 & 0 & 2 & 1 & 0\\
        0 & 1 & 0 & 0 & 0 & -1 & 0\\
        0 & 0 & 1 & 0 & 3 & 1 & 0\\ 
        0 & 0 & 0 & 1 & -1 & -1 &0\\
        0 & 0 & 0 & 0 & 0 & 0 & 0
        \end{bmatrix}$$

    \item[2.b)] \textbf{Find a maximal linearly independent set $B′_1$ of vectors from $B_1$, and show that vectors from $B_1$ that are NOT in $B′_1$ set are contained in the span of $B′_1$ (and hence, that span $B_1 =$ span $B′_1$).}
        
   Theorem 1.2.2 tells us that a set is linearly independent IFF none of the vectors in B can be written as a linear combination of the others. Using MATLAB, I checked that the system of $c_1w_1 + c_2w_2 + c_3w_3 + c_4w_4 = w_5 \text{ OR } w_6$ is consistent. Therefore, we can conclude that this is the maximal independent set $B'_1 = \{w_1, w_2, w_3, w_4\}$
        
    \item[2.c)] \textbf{What is the dimension of Span $B_1$?} The dimension of the Span $B_1$ is the number of vectors in the maximal linearly independent set $B'_1$. In this case, there are 4 independent vectors in $B'_1$ and the dimension is 4.
        
    \item[2.d)] \textbf{Find all the vectors in the intersection Span B1 $\cap$ Span B2.} I subtracted the first equation from the second equation and concatenated them into an augmented matrix. After doing a row reduction, I was able to determine that the system was consistent. Now, substituting the RREF values back into $B'_1$, I was able to get the following general form of the intersection:
    
    $$ B'_1 \cap B'_2 = s\begin{bmatrix}
    -7/9 \\ -5/9 \\ -4/3 \\ -1 \\ -4/3
    \end{bmatrix} + t\begin{bmatrix}
    -4/9 \\ 29/18 \\  11/12 \\ 3/4 \\ 5/12
    \end{bmatrix}
    + p\begin{bmatrix}
    -2/9 \\ 5/9 \\ -7/6 \\ 3/2 \\ -1/6
    \end{bmatrix}
    $$
        
    \item[2.e)] \textbf{Find the dimension of the intersection and show that this is a subspace of R5.}
    The dimension of the intersection is 3. The maximal linearly independent set of vectors $B'_1$ and $B'_2$ are both subsets of the spans $B_1$ and $B_2$. In other words, the spans of $B_1\cap B_2 = B'_1 \cap B'_2$. By definition, spans are all linear combinations of a set of vectors and constitute a subspace because they contain the zero-vector, closed under addition, and closed under scalar multiplication. Because the intersection of $B'_1 \cap B'_2$ is a span of vectors, it is a subspace.
\end{enumerate}
\end{document}
